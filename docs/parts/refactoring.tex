\part{Refactoring}
Es wurden im Laufe des Projekts verschiedene Refactorings durchgeführt.
Viele waren eher kleinerer Natur, doch einige haben auch größere Änderungen
mit sich gebracht.
In den folgenden Abschnitten wird jeweils ein Problem beschrieben und
gezeigt, wie es durch eine Anpassung der Programmstruktur gelöst werden kann.

\section{Lange Parameterliste}
Um einen Wegfinde-Algorithmus zu verwenden, wurden zuvor die
drei Parameter \inlinecode{grid}, \inlinecode{startPosition} und
\inlinecode{searchDiagonal} benötigt. Manche Verfahren verwendet zusätzlich außerdem
noch die Zielposition. Der folgende Quelltext zeigt einen Ausschnitt aus alten der
\inlinecode[\colorClasses]{Pathfind\-ingService} Klasse
(vgl. \autoref{code:pathfinding-service} für die neue Implementierung).
\begin{lstlisting}[caption={Die alte \lstinline{PathfindingService} Klasse},
    label={code:use-pathfinding}]
internal class PathfindingService : IPathfindingService
{
    public PathfindingResult @\colorMethods Dijkstra@(GridNode[][] grid,
        Position startPosition, bool searchDiagonal)
    {
        return new Dijkstra(GetSearchType(searchDiagonal))
            .ShortestPath(grid, startPosition);
    }

    private static IGetNeighbors GetSearchType(bool searchDiagonal) =>
        searchDiagonal
            ? new GetNeighborsDiagonal()
            : new GetNeighborsHorizontal();
}
\end{lstlisting}
Um die Parameterliste zu verkürzen und die Wegfinde-Algorithmen
einheitlich zu halten, wurden die verschiedenen Eigenschaften in der Klasse
\inlinecode[\colorClasses]{Grid} zusammengefasst.
Die Polymorphie der \inlinecode[\colorMethods]{GetNeighbors} Methode
ist jetzt nicht mehr über das Interface \inlinecode[\colorInterfaces]{IGetNeighbors}
gegeben, sondern über die Subklassen der abstrakte Klasse
\inlinecode[\colorClasses]{Grid} (vgl. Liskov Substitution Principle \ref{sec:lsp}).
Die Änderungen wurden in
\href{https://github.com/JensDll/pathfinding-visualization/pull/148}{Pull Request \#148}
durchgeführt.

\section{Divergierende Änderungen}
Besonders zu Beginn der Entwicklung gab es noch einige Klassen
im Projekt mit mehr als einer oder unklarer Aufgabe.
Ein Beispiel ist die \lstinline{AlgorithmService} Klasse,
welche im folgenden Quelltext zu sehen ist.
\newpage
\begin{lstlisting}[caption={{\lstinline{AlgorithmService}} Klasse},
    label={code:alg-service}]
public class AlgorithmService : IAlgorithmService
{
    public List<GridNode> GetNeighbors(GridNode[][] grid,
        (int row, int col) point)
    {
        // ...
    }

    public List<GridNode> @\colorMethods GetNeighborsDiagonal@(GridNode[][] grid,
        (int row, int col) point)
    {
        // ...
    }

    public void ConstructShortestPath(GridNode node, List<GridNode> shortestPath)
    {
        // ...
    }
}
\end{lstlisting}
Die Klasse besitzt drei Methoden, wobei diese nicht unbedingt etwas miteinander
zu tun haben und wahrscheinlich besser aufgehoben
wären in einer eigenen Entität mit verwandter Bedeutung. Die Methoden,
um benachbarte Knoten zu bestimmen, wurden wie in \ref{sec:lsp}
bereits beschrieben, in die Klasse \textbf{Grid} ausgelagert.
Da für die Berechnung des kürzesten Wegs immer ein Ausgangsknoten benötigt wird, wurde diese
Funktion als Teil der \textbf{GridNode}
Entität aufgenommen (vgl. Information Expert \ref{sec:info-expert}).
Die zu der Klasse hinzugefügten Methoden sind in \autoref{code:grid-node-entity} zu sehen.
\newpage
\begin{lstlisting}[caption={\textbf{GridNode} Entität}, label={code:grid-node-entity}]
public class GridNode
{
    // Eigenschaften der Klasse ...

    public List<GridNode> ConstructShortestPath()
    {
        var shortestPath = new List<GridNode>();
        ConstructShortestPathImpl(this, shortestPath);
        return shortestPath;
    }

    private void ConstructShortestPathImpl(GridNode node, List<GridNode> shortestPath)
    {
        if (node == null) return;
        ConstructShortestPathImpl(node.PreviousGridNode, shortestPath);
        shortestPath.Add(node);
    }
}
\end{lstlisting}
Die Änderungen wurden in
\href{https://github.com/JensDll/pathfinding-visualization/pull/103}{Pull Request \#103}
und \href{https://github.com/JensDll/pathfinding-visualization/pull/148}{\#148}
durchgeführt.