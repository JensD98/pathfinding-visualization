\part{Domain Driven Design}
Die Domänensprache des Projekts umfasst die folgenden Begriffe:
\begin{itemize}
  \item \textbf{Grid} $\sim$ Das Gitter auf dem der kürzeste Weg gesucht und
        angezeigt wird.
  \item \textbf{GridNode} $\sim$ Knoten (engl. \textit{node}),
        welcher eine Position auf dem Gitter beschreibt. Das
        Gitter besteht aus mehreren Knoten. Ein Knoten hat neben
        primitiven Werten (wie Gewicht), außerdem die
        folgenden Eigenschaften:
        \begin{itemize}[topsep=0pt]
          \item \textbf{GridNodeType} $\sim$ Der Typ des Knotens mit Werten
                wie Start, Ziel oder Wand.
          \item \textbf{Position} $\sim$ Die Koordinate des Knotens in der
                Form $(Zeile,Spalte)$, engl. $(row,column)$.
        \end{itemize}
\end{itemize}
Die meisten Begriffe wie und \textbf{GridNode} und \textbf{Position}
werden im Programmcode als Entitäten bezeichnet. Die
Wegfinde-Algorithmen sind Teil des \textit{abstraction code} auf
der Domänenebene. Sie werden als Teil eines \textit{service}
zusammengefasst mit dem Namen \texttt{Path\-findingService} und
über diesen aufgerufen.